%\documentclass[xcolor=table,handout,compress]{beamer}
\documentclass[xcolor=table,aspectratio=169]{beamer}
%--------------------------------------------------------------------------
% Common packages
%--------------------------------------------------------------------------
\usepackage[english]{babel}
\usepackage{pgfpages} % required for notes on second screen
\usepackage{graphicx}
\usepackage{subfigure}
\usepackage{multicol}
\usepackage[normalem]{ulem}

\usepackage{tabularx,ragged2e}
\usepackage{booktabs}
\usepackage{marvosym}

\makeatletter
\let\beamer@writeslidentry@miniframeson=\beamer@writeslidentry
\def\beamer@writeslidentry@miniframesoff{%
  \expandafter\beamer@ifempty\expandafter{\beamer@framestartpage}{}% does not happen normally
  {%else
    % removed \addtocontents commands
    \clearpage\beamer@notesactions%
  }
}
\newcommand*{\miniframeson}{\let\beamer@writeslidentry=\beamer@writeslidentry@miniframeson}
\newcommand*{\miniframesoff}{\let\beamer@writeslidentry=\beamer@writeslidentry@miniframesoff}
\makeatother


%--------------------------------------------------------------------------
% Load theme
%--------------------------------------------------------------------------
\usetheme{hri}

\usepackage{tikz}
\usetikzlibrary{patterns,shapes,fpu,fit,calc,mindmap,backgrounds,positioning,svg.path}


\tikzset{
  invisible/.style={opacity=0},
  visible on/.style={alt={#1{}{invisible}}},
  alt/.code args={<#1>#2#3}{%
    \alt<#1>{\pgfkeysalso{#2}}{\pgfkeysalso{#3}} % \pgfkeysalso doesn't change the path
  },
}

\newcommand*\circled[1]{\tikz[baseline=(char.base)]{
            \node[shape=circle,draw,inner sep=1pt] (char) {\bf\tiny #1};}}


%% Neat trick to have only one navigation bullet per subsection
%% http://tex.stackexchange.com/questions/64333/one-navigation-bullet-per-subsection-with-subsection-false-in-custom-beamer-them
%\usepackage{etoolbox}
%\makeatletter
%\patchcmd{\slideentry}{\advance\beamer@xpos by1\relax}{}{}{}
%\def\beamer@subsectionentry#1#2#3#4#5{\advance\beamer@xpos by1\relax}%
%\makeatother
%%%%%%%%%%%%%%%%%%%%%%%%%%%%%%%%%%%%%%%

\graphicspath{{figs/}}

% for model of anthopomorphism
\newcommand{\IPA}{{$\mathcal{A}_0$~}}
\newcommand{\SLA}{{$\mathcal{A}_\infty$~}}
\newcommand{\sla}{{\mathcal{A}_\infty}}
\newcommand{\AntMax}{{$\mathcal{A}_{max}$~}}
\newcommand{\antMax}{{\mathcal{A}_{max}}}

% for HATP plans
\newcommand{\hatpaction}[3]{#1\\\textsf{\scriptsize #2,}\\\textsf{\scriptsize #3}}
\newcommand{\stmt}[1]{{\footnotesize \tt  #1}}

% for mutual modelling
\newcommand{\Mmodel}[3]{{\mathcal{M}(#1, #2, #3)}}
\newcommand{\model}[3]{{$\mathcal{M}(#1, #2, #3)$}}
\newcommand{\Model}[3]{{$\mathcal{M}^{\circ}(#1, #2, #3)$}}

% typeset logical concept
\newcommand{\concept}[1]{{\scriptsize \texttt{#1}}}

\newcommand{\backbutton}{\hfill\hyperlink{appendix}{\beamerreturnbutton{Supplementary material}}}
%--------------------------------------------------------------------------
% General presentation settings
%--------------------------------------------------------------------------
\title{\Large Understandability}
\subtitle{Some food for thought in the context of HRI}
\date{{\bf Responsible Robotics} | Dagstuhl Seminar, 11-15 Sep 2023}
\author{Séverin Lemaignan}
\institute{{\bf PAL Robotics} Head of Social Robotics, Senior Scientist AI \& HRI}

%--------------------------------------------------------------------------
% Notes settings
%--------------------------------------------------------------------------
%\setbeameroption{show notes on second screen}
%\setbeameroption{hide notes}

\begin{document}


%%%%%%%%%%%%%%%%%%%%%%%%%%%%%%%%%%%%%%%%%%%%%%%%%%%%%%%%


%%%%%%%%%%%%%%%%%%%%%%%%%%%%%%%%%%%%%%%%%%%%%%%%%%%%%%%%

\licenseframe{github.com/severin-lemaignan/2023-understandability-dagstuhl}

\maketitle

\imageframe[color=black,caption=www.unesco.org/en/digital-education/ed-tech-tragedy]{ed-tech-tragedy-unesco}

\imageframe{my_background/bg-map}



\begin{frame}<2-7>[label=soro]{Social Robotics}
    \centering

    \begin{columns}
        \begin{column}{0.7\linewidth}

            \only<1>{
                Creating interactive robots that are \textbf{embedded and
                understand their (human) social context}; \textbf{generate and adopt appropriate
                social behaviours}; have a \textbf{positive impact on human
                society}.

                \vspace{2em}
                $\Rightarrow$ designing and implementing the \textbf{assistant and
                companion robots} for  tomorrow.

                \vspace{2em}
                $\Rightarrow$ direct impact on ageing society, education, customer service;
                \textbf{major socio-economic
                challenge}

                \vspace{2em}
                $\Rightarrow$ but also: \textbf{unique opportunity to provide individualised
                support for vulnerable populations}
            }

            \only<2->{
                \textbf{Major scientific challenges}:

                \begin{itemize}
                    \item<2-> Model open-ended, underspecified situations; rich
                        semantics; complex social dynamics;
                    \item<3-> Close the interaction loop;
                    \item<4-> Understand and sustain long-term autonomous social interactions;
                    \item<5-> Real-world algorithmic robustness;
                    \item<6-> Complex ethical landscape \\ (correct, ethical =
                        moving, context-based target);
                    \item<7-> $\Rightarrow$ cross-disciplinary \& holistic approach required
                    \item<7-> $\Rightarrow$ involve all the stakeholders;
                        participatory approach

                \end{itemize}


                \uncover<8>{
                \vspace{1em}


                \setbeamercolor{hriSec1Demo}{bg=hriSec3Dark,fg=white}
                \begin{beamercolorbox}[wd=\linewidth,ht=8ex,dp=0.7ex]{hriSec1Demo}
                    \large \bf \centering
                    Socially-Driven Autonomous Robots \\
                    for \\
                    Real-world Human-Robot Interactions
                \end{beamercolorbox}
            }


            }
        \end{column}
        \begin{column}{0.35\linewidth}
            \begin{center}
                \includegraphics[trim=15cm 0 11cm 0,clip,width=0.7\linewidth]{ranger/ranger_funny_glasses}
            \end{center}
        \end{column}
    \end{columns}

    \toplogo{logo_bw}
\end{frame}

\section*{Understandability}

\begin{frame}{Understandability}

    ...`Understandability'?

    \pause

    In EU's \href{https://digital-strategy.ec.europa.eu/en/library/ethics-guidelines-trustworthy-ai}{Ethics Guidelines
    for Trustworthy AI}, keywords like:

    \begin{itemize}
        \item transparency
        \item explainability (eg XAI)
        \item tracability (eg Winfield's \emph{ethical black box})
    \end{itemize}

    \pause

    Useful mechanistic approach, ...but does it capture well the \emph{feeling}
    of `understanding' the machine? (ie, in-situ, in-the-moment, intuitive
    understanding)

    \pause

    In other words, would eg explainability and tracability by themselves be
    sufficient to establish trust? {\footnotesize (trust \emph{in bridge} vs trust \emph{in
    people}?)}

    \toplogo{logo_bw}
\end{frame}
%
%\begin{frame}{Do you trust public research funding?}
%
%    ``Getting a project funded is really like a lottery!''
%
%    \pause
%
%    Do you understand the system? ...often yes (...we are part of it!)
%
%    However, results are hard to predict, seem at time partially random
%
%\end{frame}

\begin{frame}{Do we actually really want to understand our robots?}

    Users (ourselves included) want technology that `just work'; cf Obadia's idea of the inherent \emph{magic} of technology

    \pause

    Example: when people ask something to ChatGPT, they might complain that the
    result is incorrect (hallucinated or otherwise) $\Rightarrow$ they
    want the visible, \emph{surface} result to fixed.

    \pause

    Do they want to \emph{understand} why ChatGPT got it wrong? I'd argue that
    the vast majority do not really care. {\footnotesize Also, psychological
    angle, eg Nass' Media Equation}

    \pause

    However, I would also argue that, to trust the system, what we really want
    is to be able to \alert<5>{confidently}\bubblemark{mark1} build a
    \alert<6>{robust mental model}\bubblemark{mark2}, with a strong \bubblemark{mark3}\alert<7>{predictive
    power}.

%    \bubble<5>[200][0.5][6cm]{mark1}{extrapolating from known mental models;
%    building the model implicitly while interacting;...}
%    \bubble<6>[220][0.5][5cm]{mark2}{once established, the model remains
%    essentially stable and
%    valid over time}
%    \bubble<7>[300][0.5][5cm]{mark3}{this foster the resulting trust}

    \toplogo{logo_bw}
\end{frame}

\begin{frame}{}

- When did The New York Times first report on "artificial intelligence"?

- I'm unable to provide specific dates or historical details as my knowledge is limited to information available up until September 2021. However, I can tell you that The New York Times has been reporting on artificial intelligence for many years, with coverage dating back to at least the mid-20th century when the concept of AI first emerged. To find the exact date of their first report on artificial intelligence, I recommend searching The New York Times archives or contacting their research department for more specific historical information.
\end{frame}

{
\paper{Lemaignan, Fink, Dillenbourg, Braboszcz {\bf The Cognitive Correlates of Anthropomorphism}, HRI 2014}
\begin{frame}{Cognitive Interpretation?}

    \vspace{-1em}
    \begin{figure}
        \centering
        \resizebox{0.9\paperwidth}{!}{
            \begin{tikzpicture}
                \baselineskip=8pt

                \path[fill=hriSec1Comp!50] (0,0) rectangle (10.5,2);
                \path[fill=hriSec1!50] (0,2) rectangle (10.5,4);
                \path[fill=hriSec3Comp!50] (0,4) rectangle (10.5,6);

                \draw (-0.3,1) node[rotate=90] {Stage 1}
                (-0.3,3) node[rotate=90] {Stage 2}
                (-0.3,5) node[rotate=90] {Stage 3};

                \draw[->] (0,0) -- (11,0) node[anchor=north] {$t$};
                \draw[-|] (0,0) -- (0,6.5) node[anchor=east] {};
                % Us
                \draw[ultra thick, ->] (0,0) -- (1,2) -- (3,2) -- (4,4) -- (6,4) -- (7,6) --
                (9.5,6);
                \draw[ultra thick, dashed, ->] (3,2) -- (9.5,2);
                \draw[ultra thick, dashed, ->] (6,4) -- (9.5,4);

                \draw (11,1) node[align=left, anchor=west]{\scriptsize pre-cognitive\\\scriptsize anthropomorphism}; %label

                \draw (11,3) node[align=left, anchor=west] {\scriptsize{projection of existing}\\\scriptsize{mental models}\\\scriptsize{\it (familiarity)}}; %label

                \draw (11,5) node[align=left, anchor=west] {\scriptsize{adapted mental
                model} \\ $\to$ \scriptsize{adapted interaction}\\\scriptsize{\it (anthropomorphic or not)}}; %label


                \draw[dashed,->] (1,0.1) to[bend right] (2,1.9)  node at (4.5,1) {\tiny{\it observation (shape, motion, sound)}}; %label

                \draw[dashed,->] (4,2.1) to[bend right] (5,3.9)  node[align=left] at (7.5,3) {\tiny \it observation (interactive behavior) \\ \tiny \it or short interaction}; %label

                \draw[dashed,->] (7,4.1) to[bend right] (8,5.9)  node[align=left] at (9,5)
                {\tiny \it contextualized\\ \tiny \it interaction}; %label

            \end{tikzpicture}
        }
    \end{figure}

    \badge[caption=Julia Fink]{colleagues/julia}
    %\toplogo{logo_bw}
\end{frame}
}

\begin{frame}{What means `responsible' in that context?}
    If we adopt the cognitive perspective of \emph{understanding} the robot as
    the \textbf{completion of a robust mental model} of the machine, where does
    that leave us wrt. \emph{responsible} understandability?

    \pause

    $\Rightarrow$ design robots' behaviours to foster the creation of detailed
    and correct mental models

    \pause

    ...or even let the end-users co-design the robot's cognitive capabilities

    \toplogo{logo_bw}
\end{frame}

\section*{`End-to-end' co-design}

\imageframe[color=black,caption=End-to-end co-design: gym coach]{couch25k/hri-169.jpg}

{
    \paper{Winkle et al. \textbf{In-Situ Learning from a Domain Expert for Real World Socially Assistive Robot Deployment} RSS 2020}

\begin{frame}{Couch-to-5k study}

    \vspace{0.2cm}

    \begin{columns}
        \begin{column}{0.5\linewidth}
                \centering
                \includegraphics[height=3cm]{couch25k/hri.jpg}
        \end{column}
        \begin{column}{0.5\linewidth}
                \centering
                \includegraphics[height=3cm]{couch25k/supervised.jpg}
        \end{column}
    \end{columns}

    \begin{itemize}
        \item 9 participants
        \item 3 months; 27 one-hour sessions per participants
        \item $|state| = 20$; $|action\_space| = 11$
        \item Includes participants' personality (Big-5) as input feature
    \end{itemize}

    %\toplogo{logo_bw}
    \badge[caption=Katie Winkle]{colleagues/katie}
\end{frame}
}

{
    \paper{Winkle et al. \textbf{In-Situ Learning from a Domain Expert for Real World Socially Assistive Robot Deployment} RSS 2020}

\begin{frame}{Co-design for real-world + long-term}

    \begin{center}
        \resizebox{!}{0.75\pageheight}{
        \begin{tikzpicture}
            \node at (0,0) {\includegraphics[width=0.6\linewidth]{figs/couch25k/couch25km-mock.jpg}};
           \node[anchor=north] at (7,0.5) {\includegraphics[trim=0 15cm 0 0,clip,width=0.6\linewidth]{figs/couch25k/codesign.jpg}};
           \node[anchor=north] at (-3.5,-1.5) {\includegraphics[width=0.5\linewidth]{figs/couch25k/teacher-ui.png}};
        \end{tikzpicture}
    }

    \end{center}
    %\toplogo{logo_bw}
    \badge[caption=Katie Winkle]{colleagues/katie}
\end{frame}
}


\imageframe[scale=0.9]{DesignProcess-base}
\imageframe[scale=0.9]{DesignProcess-irl}
\imageframe[scale=0.9]{DesignProcess-full}



{
    \paper{Winkle et al. \textbf{In-Situ Learning from a Domain Expert for Real World Socially Assistive Robot Deployment}
    RSS 2020}
    \begin{frame}{Learnt policies}
        \begin{center}
            \includegraphics<1>[width=0.9\linewidth]{figs/couch25k/finalactiondist-no-autonomous.png}
            \includegraphics<2>[width=0.9\linewidth]{figs/couch25k/finalactiondist.png}
            \includegraphics<3>[width=0.9\linewidth]{figs/couch25k/fullcomp_lbmr.pdf}
        \end{center}
    \toplogo{logo_bw}
\end{frame}
}


{
    \paper{Senft et al. \textbf{SPARC: Supervised progressively autonomous
robot competencies}, ICSR 2015;\newline
    Winkle et al. \textbf{LEADOR: End-to-End Participatory
    Design of Autonomous Social Robots} FrontiersIn AI \& Robotics 2021}
\begin{frame}{LEADOR: end-to-end co-design}

    \begin{columns}
        \begin{column}{0.6\linewidth}
    \begin{itemize}
        \item a successful technical solution to replace the wizard;
        \item but equally important, a \textbf{end-to-end} participatory design
            methodology (\emph{LEADOR: Led-by-Experts Automation \\and Design Of
            Robots})
    \end{itemize}



        \end{column}
        \begin{column}{0.4\linewidth}
            \begin{center}

               % \includegraphics<1>[width=\columnwidth]{DesignProcess-base}
               % \includegraphics<2>[width=\columnwidth]{DesignProcess-irl}
                \includegraphics[width=\columnwidth]{DesignProcess-full}
            \end{center}
        \end{column}
    \end{columns}

    %\toplogo{logo_bw}
    \badge[caption=Emmanuel + Katie]{colleagues/emmanuel_katie}
\end{frame}
}

\begin{frame}{}

    More importantly,

    \vspace{2em}

    \begin{center}

    \large{``It was like training my assistant. Now I let it start with the
    next participant, while I do stretching with the previous one.''}

    \end{center}

    \vspace{2em}

    \pause

    ...does he \emph{understand} the robot? \footnotesize{yes and no}

    \pause

    \normalsize ...is it \emph{Responsible Robotics\textcopyright}?

    \toplogo{logo_bw}
\end{frame}


\section*{Very brief look at the industry}

\begin{frame}{Does it translate to industry?}
    \begin{columns}
        \begin{column}{0.7\linewidth}
            \only<1-2>{
            Can we turn the LEADOR paradigm into a generic \& robust enough technique for broad usage?
            }

            \only<2>{
                \vspace{1cm}
                \textbf{Two key use-cases for PAL Robotics}:

            \begin{itemize}
                \item robots that know when to help in public administrations
                \item elderly care and isolation (but... \textbf{rHHI}:
                    robot-supported Human-Human Interactions!)
            \end{itemize}
            }

        \end{column}
        \begin{column}{0.3\linewidth}
            \begin{center}
                \includegraphics[height=0.7\pageheight]{figs/aris.jpg}
            \end{center}
        \end{column}
    \end{columns}

    \toplogo{logo_bw}

\end{frame}

\imageframe[caption=NHoA: Never HOme Alone]{nhoa/miro-interviews}
\imageframe[caption=H2020 SHAPES]{shapes/visitsmallorca}
\imageframe[color=black, caption=AMIBA]{amiba/amiba-1}
\imageframe[color=black, caption=AMIBA]{amiba/amiba-2}
\imageframe[color=black, caption=AMIBA]{amiba/amiba-3}

\begin{frame}{Does it translate to industry?}
    \begin{columns}
        \begin{column}{0.7\linewidth}
            \begin{itemize}
                \item<+-> practical issues are typically overlooked (co-design
                    in-situ)
                 \item<+-> adoption: primarily from \emph{secondary} users (carers,
                    teachers...)
                \item<+-> needs are not always what we think (co-design with all
                    stakeholders)
                \item<+-> robots do not live in isolation: think 'eco-systems'
                    and embrace ``mis-use"/mis-understanding of the robot
                \item<+-> from interaction to habits: (individualised) long-term
                    \emph{mutual shaping}
                \item<+-> importance of the regulatory framework (legal +
                    ethical) to steer an
                    otherwise `engineering-focused' mindset

            \end{itemize}
        \end{column}
        \begin{column}{0.3\linewidth}
            \begin{center}
                \includegraphics[height=0.7\pageheight]{figs/aris.jpg}
            \end{center}
        \end{column}
    \end{columns}

    \toplogo{logo_bw}

\end{frame}

\imageframe[scale=0.9]{robots_for_care/stakeholders}

\section*{Final remarks and summary}

\begin{frame}{Some final remarks}

    \only<1>{
        Co-design requires bootstrapping: you need to understand at
            \emph{some} level the physical (easy) and cognitive (much harder)
            capabilities of the robot to `invent' its role

            $\rightarrow$ iterative process

            $\rightarrow$ slow process

            \footnotesize{(though more of a problem for traditional co-design than lifelong,
            end-to-end co-design)}

        }

    \only<2>{
    Sabanovic's \emph{mutual shaping} is fundamental and not yet fully
    internalised by researchers/engineers

    $\rightarrow$ you (re)shape your relation to -- and understanding of -- the robot by using it

    $\Rightarrow$ \emph{Understanding} the
    robot likely means that the robot also has to understand you!

    \footnotesize{(here, the meaning of \emph{understand} is left as an exercice
    to the reader)}
    }

    \only<3>{
        ...as such:

        can we really separate the human's understanding of the robot, from
        the robot's understanding of the human, of its environment, and... of
        itself? (introspection)
    }
\end{frame}

\begin{frame}{Anecdote: What if we ask the system to explain its behaviour?}

    \only<1-2>{
    \texttt{Sally (a girl) has 3 brothers. Each brother has 2 sisters. How many
    sisters does Sally have?}

    \pause

    GPT-4:

    \texttt{Sally has 2 sisters.}
}

    \only<3-4>{

    \texttt{Sally (a girl) has 3 brothers. Each brother has 2 sisters. How many
    sisters does Sally have?} \textbf{\texttt{Let's think step by step.}}

    \onslide<4>{

    GPT-4:

    \texttt{Sally has 3 brothers. Each of these brothers has 2 sisters. This means that there are 2 girls in the family, including Sally. Therefore, Sally has 1 sister.}

    (incidentally, \href{https://benchmarks.llmonitor.com/}{GPT-4 is the only model out of 60 tested that get the
    correct answer})
}
    }
 \toplogo{logo_bw}
\end{frame}


\begin{frame}{Summary}
    \begin{itemize}
        \item  \textbf{analytic understanding} vs \textbf{intuitive
            understanding}: \emph{understand} has a very different meaning for the
            regulator/designer, and for the non-expert end-user
        \item \textbf{lifelong responsibility on shifting sands}:
            interaction is reshaped over time by usage, accommodation,
            individual personality traits; responsibility does not stop at the
            point of sale
        \item \textbf{robots exist in whole ecosystems}: multiple
            stakeholders, potentially changing stakeholders, with wildly
            different needs and goals
    \end{itemize}
\end{frame}

\section*{3 case-studies}

\begin{frame}{Trustworthy AI checklist}
    \begin{columns}
        \begin{column}{0.6\linewidth}

    \begin{itemize}
        \item human autonomy, human agency
        \item consent
        \item human oversight, especially if the system evolves (ML)
        \item technical robustness and safety
        \item respect for privacy, data ownership
        \item traceability, explainability, communication
        \item unfair bias avoidance, stakeholder participation
        \item sustainability
        \item social impact
        \item Accountability, documenting trade-offs

    \end{itemize}
        \end{column}
        \begin{column}{0.4\linewidth}
            \begin{center}
                \includegraphics[width=0.8\columnwidth]{eu-ethics-guidelines}
            \end{center}
        \end{column}
    \end{columns}
\end{frame}

% Eco-system 1: stockbot and retail
\imageframe[color=black]{ecosystems/decathlon-stockbot}
\imageframe[scale=0.9]{ecosystems/stockbot}

% Eco-system 2: gym coach
\imageframe[color=black]{couch25k/hri-169.jpg}
\imageframe[scale=0.9]{ecosystems/gym-coach}
\imageframe[scale=0.9]{ecosystems/gym-coach-power}


% Eco-system 3: surveillance, healthcare
\imageframe[color=black]{ecosystems/ari-healthcare}
\imageframe[scale=0.9]{ecosystems/healthcare-ari}

%%%%%%%%%%%%%%%%%%%%%%%%%%%%%%%%%%%%%%%%%%%%%%%%%%%%%%%%
{
    \fullbackground[color=black]{thank_you}
    \begin{frame}[plain]
    \end{frame}
}

\end{document}
